\documentclass[12pt,a4paper,twoside,openright]{book}

% ===== PACKAGES =====
\usepackage[utf8]{inputenc}
\usepackage[T1]{fontenc}
\usepackage[french]{babel}
\usepackage{geometry}
\usepackage{fancyhdr}
\usepackage{titlesec}
\usepackage{tocloft}
\usepackage{graphicx}
\usepackage{xcolor}
\usepackage{hyperref}
\usepackage{amsmath,amssymb}
\usepackage{listings}
\usepackage{enumitem}
\usepackage{float}
\usepackage{caption}
\usepackage{subcaption}
\usepackage{booktabs}
\usepackage{longtable}
\usepackage{microtype}
\usepackage{setspace}

% ===== CONFIGURATION GÉOMÉTRIE =====
\geometry{
    top=2.5cm,
    bottom=2.5cm,
    left=3cm,
    right=2.5cm,
    headheight=15pt,
    headsep=1cm,
    footskip=1cm
}

% ===== COULEURS =====
\definecolor{primarycolor}{RGB}{0,102,204}
\definecolor{secondarycolor}{RGB}{102,102,102}
\definecolor{accentcolor}{RGB}{255,102,0}

% ===== CONFIGURATION HYPERREF =====
\hypersetup{
    colorlinks=true,
    linkcolor=primarycolor,
    filecolor=primarycolor,
    urlcolor=primarycolor,
    citecolor=primarycolor,
    pdftitle={Titre du Livre},
    pdfauthor={Nom de l'Auteur},
    pdfsubject={Sujet du livre},
    pdfkeywords={mots-clés},
    bookmarksnumbered=true,
    bookmarksopen=true
}

% ===== STYLE DES TITRES =====
\titleformat{\chapter}[display]
{\normalfont\huge\bfseries\color{primarycolor}}
{\chaptertitlename\ \thechapter}{20pt}{\Huge}

\titleformat{\section}
{\normalfont\Large\bfseries\color{primarycolor}}
{\thesection}{1em}{}

\titleformat{\subsection}
{\normalfont\large\bfseries\color{secondarycolor}}
{\thesubsection}{1em}{}

% ===== EN-TÊTES ET PIEDS DE PAGE =====
\pagestyle{fancy}
\fancyhf{}
\fancyhead[LE]{\leftmark}
\fancyhead[RO]{\rightmark}
\fancyfoot[LE,RO]{\thepage}
\renewcommand{\headrulewidth}{0.4pt}
\renewcommand{\footrulewidth}{0pt}

% Style pour les pages de début de chapitre
\fancypagestyle{plain}{
    \fancyhf{}
    \fancyfoot[C]{\thepage}
    \renewcommand{\headrulewidth}{0pt}
}

% ===== CONFIGURATION LISTINGS (CODE) =====
\lstset{
    basicstyle=\ttfamily\small,
    backgroundcolor=\color{gray!10},
    frame=single,
    rulecolor=\color{gray!30},
    numbers=left,
    numberstyle=\tiny\color{gray},
    keywordstyle=\color{primarycolor}\bfseries,
    commentstyle=\color{secondarycolor}\itshape,
    stringstyle=\color{accentcolor},
    breaklines=true,
    breakatwhitespace=true,
    tabsize=4,
    showspaces=false,
    showstringspaces=false
}

% ===== ESPACEMENT =====
\onehalfspacing

% ===== INFORMATIONS DU LIVRE =====
\title{{\Huge\textbf{TITRE DU LIVRE}}\\[0.5cm]
       {\Large Sous-titre si nécessaire}}
\author{{\Large Nom de l'Auteur}\\[0.3cm]
        {\normalsize Titre/Affiliation}}
\date{\today}

% ===== DÉBUT DU DOCUMENT =====
\begin{document}

% ===== PAGE DE TITRE =====
\frontmatter
\begin{titlepage}
    \centering
    \vspace*{2cm}
    
    {\Huge\textbf{\textcolor{primarycolor}{TITRE DU LIVRE}}\par}
    \vspace{1cm}
    {\Large Sous-titre explicatif\par}
    \vspace{2cm}
    
    {\Large\textbf{Nom de l'Auteur}\par}
    \vspace{0.5cm}
    {\normalsize Titre ou Affiliation\par}
    
    \vfill
    
    % Logo ou image si nécessaire
    % \includegraphics[width=0.3\textwidth]{logo.png}
    
    \vfill
    
    {\large Édition \today\par}
\end{titlepage}

% ===== PAGE DE COPYRIGHT =====
\newpage
\thispagestyle{empty}
\vspace*{\fill}
\begin{center}
    \textcopyright\ \the\year\ Nom de l'Auteur\\[0.5cm]
    Tous droits réservés. Aucune partie de cette publication ne peut être reproduite,\\
    stockée dans un système de récupération, ou transmise sous quelque forme\\
    ou par quelque moyen que ce soit, électronique, mécanique, photocopie,\\
    enregistrement ou autre, sans l'autorisation écrite préalable de l'auteur.\\[1cm]
    
    ISBN: 000-0-00-000000-0\\[0.5cm]
    
    Première édition: \today\\[1cm]
    
    Imprimé en France
\end{center}
\vspace*{\fill}

% ===== DÉDICACE (OPTIONNEL) =====
\newpage
\thispagestyle{empty}
\vspace*{\fill}
\begin{center}
    \textit{À mes proches,\\
    pour leur soutien inconditionnel.}
\end{center}
\vspace*{\fill}

% ===== PRÉFACE =====
\chapter*{Préface}
\addcontentsline{toc}{chapter}{Préface}

Voici la préface de votre livre. Expliquez le contexte, les motivations qui vous ont poussé à écrire ce livre, et ce que le lecteur peut en attendre.

% ===== REMERCIEMENTS =====
\chapter*{Remerciements}
\addcontentsline{toc}{chapter}{Remerciements}

Je tiens à remercier toutes les personnes qui ont contribué à la réalisation de ce livre...

% ===== TABLE DES MATIÈRES =====
\tableofcontents

% ===== LISTE DES FIGURES (OPTIONNEL) =====
\listoffigures

% ===== LISTE DES TABLEAUX (OPTIONNEL) =====
\listoftables

% ===== CORPS PRINCIPAL =====
\mainmatter

\chapter{Introduction}

Voici le premier chapitre de votre livre. Vous pouvez structurer votre contenu avec des sections et sous-sections.

\section{Contexte}

Décrivez le contexte de votre sujet.

\subsection{Problématique}

Présentez la problématique abordée.

\section{Objectifs}

Énumérez les objectifs de votre livre :

\begin{enumerate}
    \item Premier objectif
    \item Deuxième objectif
    \item Troisième objectif
\end{enumerate}

\chapter{Développement}

\section{Première partie}

Développez votre contenu ici.

\subsection{Exemple de figure}

Vous pouvez insérer des figures comme ceci :

\begin{figure}[H]
    \centering
    % \includegraphics[width=0.8\textwidth]{image.png}
    \caption{Légende de votre figure}
    \label{fig:exemple}
\end{figure}

\subsection{Exemple de tableau}

Et des tableaux comme ceci :

\begin{table}[H]
    \centering
    \caption{Exemple de tableau}
    \label{tab:exemple}
    \begin{tabular}{@{}lcc@{}}
        \toprule
        Élément & Valeur 1 & Valeur 2 \\
        \midrule
        A & 10 & 20 \\
        B & 15 & 25 \\
        C & 20 & 30 \\
        \bottomrule
    \end{tabular}
\end{table}

\subsection{Exemple de code}

Pour du code :

\begin{lstlisting}[language=Python, caption=Exemple de code Python]
def hello_world():
    print("Hello, World!")
    return True

if __name__ == "__main__":
    hello_world()
\end{lstlisting}

\chapter{Conclusion}

Concluez votre livre en résumant les points clés et en ouvrant sur des perspectives.

% ===== ANNEXES =====
\appendix

\chapter{Annexe A : Ressources supplémentaires}

Ajoutez ici des ressources supplémentaires, des références, etc.

% ===== BIBLIOGRAPHIE =====
\backmatter

\begin{thebibliography}{99}
    \bibitem{ref1} Auteur, A. (2023). \textit{Titre du livre}. Éditeur.
    \bibitem{ref2} Auteur, B. (2022). Article de journal. \textit{Nom du Journal}, 15(3), 123-145.
\end{thebibliography}

% ===== INDEX (OPTIONNEL) =====
% \printindex

\end{document}